\documentclass[a4paper,twocolumn,twoside,notitlepage,10pt]{article}
\usepackage[margin=2cm,columnsep=.5cm]{geometry}
\usepackage[greek,USenglish]{babel}
\usepackage[english]{isodate}
\languageattribute{greek}{ancient}% \usepackage[artemisia]{begingreek}  % https://ctan.math.illinois.edu/macros/latex/contrib/begingreek/begingreek.pdf
\usepackage{imakeidx}
%\usepackage[nopostdot,nogroupskip,nonumberlist]{glossaries-extra}
\usepackage[automake]{glossaries-extra}
\usepackage{enumitem}
\usepackage{amssymb}

\usepackage{booktabs,caption}



\title{Cognates of Cleanliness Doctrine}
\author{K. E. D. Paulsson}

\glsdisablehyper
\makeglossaries

\indexsetup{level=\section*,toclevel=section}
\makeindex[name=vicelist,title=Lists with Vices,columns=2]
\makeindex[name=eng,title=Vices in English,columns=2]
\makeindex[name=grc,title=Vices in Greek,columns=2]

\setcounter{secnumdepth}{0}


\def \tdntBertram {Bertram, Georg}

\begin{document}
\frenchspacing
\newcommand{\entlbl}[1]{\index[eng]{#1}\label{itm:#1}}
\newcommand{\entref}[1]{\emph{#1} ~($\rightarrow$~\pageref{itm:#1})}
\newcommand{\entrefgls}[1]{\emph{#1}$\rightarrow$\pageref{itm:#1}}
\newcommand{\grc}[1]{\greektext{#1}\latintext}
\newcommand{\cdfoot}[2]{\footnote{\emph{CD}, s.v. ``{#1},'' accessed \printdate{#2}.}}
\newcommand{\bkfoot}[3]{\footnote{{#3}, ``{#1},'' \emph{TDNT} {#2}.}}
\newcommand{\bksfoot}[2]{\footnote{{#2}, \emph{TDNT} {#1}.}}
%\newcommand{\cdfoot}[2]{\footnote{\emph{Cambridge Dictionary}, s.v. ``{#1},'' accessed \printdate{#2},\\https://dictionary.cambridge.org/dictionary/english/{#1}.}}


\newcommand{\entgls}[1]{\gls{#1} \glsdesc{#1}}
\newcommand{\entglos}[1]{\\\textbf{\gls{#1}} \glsdesc{#1}}



\maketitle

\section{Abstract}
Cleanliness doctrine has set its traces in the modern church teachings without heeding the true
important message. But has turned into an emotional argument without applicable meaning and
thereby misleads people to believe things that are unstable in faith.

\section{Abbreviations}
\begin{description}
    \item[NA28] Novum Testamentum Graece, Nestle-Aland, 28th ed.
	\item[NT] The New Testament of the Bible.
	\item[OT] The Old Testament of the Bible.
	\item[BW10] BibleWorks 10 bible software with report tool.
	\item[BNM] BW10 Greek New Testament Morphology.
\end{description}

\section{Introduction}

\begin{tabular}{@{}l l@{}}
  \hline
  Cognates & Antonyms \\
  \hline
  \grc{σπίλος} & \grc{ἄσπιλος} \\
  \grc{σπιλόω} &  \\
  \hline
  \grc{μιαίνω} & \grc{ἀμίαντος} \\
  \hline
  \grc{καθαρός} & \grc{ἀκάθαρτος} \\
  \grc{καθαρίζω} & \grc{ἀκαθαρσία} \\
  \grc{καθαρότης} &  \\
  \grc{καθαρισμός} &  \\
  \hline
  % & \grc{ἁγνίζω} \\
  \grc{ἁγνός} & \\
  \grc{ἁγνεία} & \\
  \grc{ἁγνότης} & \\
  \hline
  \grc{κοινός} & \\
  \grc{κοινόω} &  \\
  \hline
  \grc{μολύνω} & \\
  \grc{μολυσμός} &  \\
  \hline
  \grc{μῶμος} & \grc{ἄμωμος} \\
  & \grc{ἀμώμητος} \\
  \hline
  \grc{ῥυτίς} & X \\
  \hline
  \grc{ῥυπαρία} & \\
  \hline
\end{tabular}

\section{Vice Entries}
\begin{description}[leftmargin=0pt]
    \item[Birthmark,]
\entlbl{decent}

\grc{σπίλος, σπιλόω, ἄσπιλος}
\index[grc]{βδελυσσομαι@\grc{βδελύσσομαι}}
(\textit{spilos, spiloō, aspilos}):
\newglossaryentry{spilos, spiloō, aspilos}
{
    name=\grc{σπίλος, σπιλόω, ἄσπιλος},
    description={\entrefgls{mole}},
    sort=βδελυσσομαι@\grc{spilos, spiloō, aspilos}
} At a first glance it's impossible to understand the meanining of \grc{σπίλος}, 
according to Liddell it means \emph{a spot}, \emph{stain}, \emph{blemish} but that's all
that is mentioned. In context of the Bible, \grc{σπίλος} together with related cognates 
are used only 8 times. Also other religious Greek dictionaries give a similar insufficient
material to fully understand its true meaning. Only when we turn to modern Greek we get a
hint of what it actually means. The modern meaning is 
\emph{naevus}\footnote{\emph{TechDico}, s.v. ``\grc{σπίλος},'' accessed 2023--09--22,\\https://www.techdico.com/translation/greek-english/\grc{σπίλος}.html.},
which is presumably a strawberry birthmark\footnote{\emph{Oxford Reference}, s.v. ``naevus,'' accessed 2023--09--22,\\https://www.oxfordreference.com/search?q=naevus.}, 
possibly it could spread such as skincancer. 
Furthermore, Paul writes that Jesus wishes to ``present it to himself the assembly in glory, not having \emph{spot} [emphasis added] or wrinkle'' Eph 5:27 YLT, in this verse
the spot which is described, is a very strawberry birthmark. Also in 1 Tim 6:14 Paul admonishes Timothy in Christ Jesus to ``keep the command \emph{unspotted} [emphasis added], 
unblameable, till the manifestation of our Lord Jesus Christ,'' YLT. In this very context Paul had given a commandment to keep in v.11. Totally there are two ways to be
blemished with spiritual strawberry birthmarks in the NT. The first one is in Jas 3:6 saying that the tongue is the member that produces naevus over our whole body and is
ignited by the fire of hell. The naevus in this case should be understood spiritually and the passage should be understood properly by reading the full context.
The second contamination by strawberry birthmarks is mentioned in Jude 1:23, in this context we are admonished to show grace to those who need to be snatched out of the
fire, while being disgusted by the undergarment being blemished by the flesh, or likely the \emph{meat}. The context is very advanced and should be thoroughly
studied. The key components could be found in verse 7--8, presumably connections to sex purchase, unnatural meat and defilment in sleep with visions and nocturnal emissions.
In 2 Pet 3:14 there is a solution for those that has been blemished by the aforementioned behaviors that caused naevus. ``Wherefore beloved, while await you shall speed to be 
unblemished and blameless at piece in him (NA28; my translation).'' This gives the sinners a strong encouragment to find a cure in Christ before he comes back. While they remain
in the world, the only way to keep unblemished from the naevus of the world, are to visit widows and orphans in their tribulation as the only true religion, in accordance with Jas 1:27.
The last thing to be said is that the Blood of Christ with which we were bought is without naevus (1 Pet 1:19).

    \item[Common, commonize,]
\entlbl{common, commonize}

\grc{κοινός, κοινόω}
\index[grc]{βδελυσσομαι@\grc{βδελύσσομαι}}
(\textit{koinos, koinoō}):
\newglossaryentry{koinos, koinoō}
{
    name=\grc{κοινός, κοινόω},
    description={\entrefgls{common, commonize}},
    sort=βδελυσσομαι@\grc{κοινός, κοινόω}
}
In examining what type of cleanliness or uncleanliness that \grc{κοινός} is used for as a biblical synonym it must be
understood how it was used carnally by the Pharisees. Both in Matt 15 and in Mark 7, the Pharisees had a serious
argument why Jesus allowed the disciples to eat with \emph{unwashed hands}, that is hands that had been commonized by being
in physical touch with what was common.
Further the religious Jews accused Paul to bring Gentiles into the Temple in Jerusalem and thereby commonizing the holy
site and starting a riot, as mentioned in Acts 21:28.
Also in Acts chapter 10--11 Peter is guided by God to believe that what was unsanitary had been declared clean, no
longer commonized, as in Acts 10:28 ``But God has shown me to not call any human \emph{common} or \emph{unsanitary}
(NA28; my translation).'' God did also clearly state this in 10:15b ``Those who God \emph{sanitized}, you shall not \emph{commonize} (NA28; my translation).''
What is said in Mark 7:18--19 and Matt 15:16--18 is all about uncleanliness, Jesus clearly states that what you eat and later waste is
not what makes one common. What really makes one common is what comes out from the heart either in
actions or as encouraging babble. That which is mentioned are two sets of vices from parallel passages but with somewhat
different sins mentioned, and therefore, some sins may require to be talked about in order to make common.
The two sets of vices are:

%\begin{table}[]
%\caption{}
% \label{tab:my-table}
\begin{tabular}{lll}
& Matt 15:19 & Mark 7:21--22 \\
\grc{ψευδομαρτυρία} & \hfil \checkmark &  \\
\grc{πορνεία} & \hfil \checkmark & \hfil \checkmark \\
\grc{κλοπή} & \hfil \checkmark & \hfil \checkmark \\
\grc{φόνος} & \hfil \checkmark & \hfil \checkmark \\
\grc{μοιχεία} & \hfil \checkmark & \hfil \checkmark \\
\grc{πλεονεξία} &  & \hfil \checkmark \\
\grc{πονηρία} &  & \hfil \checkmark \\
\grc{δόλος} &  & \hfil \checkmark \\
\grc{ἀσέλγεια} &  & \hfil \checkmark \\
\grc{ὀφθαλμός πονηρός} &  & \hfil \checkmark \\
\grc{βλασφημία} &  & \hfil \checkmark \\
\grc{ὑπερηφανία} & \hfil \checkmark & \hfil \checkmark \\
\grc{ἀφροσύνη} &  & \hfil \checkmark
\end{tabular}
%\end{table}

Mark 7:15 mentions that it is what comes out of the person which defiles them, both in Matthew and in Mark according
to their respective context makes clear that it is only what comes out from the heart that defiles men and, in
Matt 15:10 specifically what comes out through the mouth, considered to be verbatim, that commonizes that person.
Simply said, it is when the person starts to speak according to the substandard reasonings of their heart, when
they actually become common.
Moreover, in Rom 14:14 we find a rule that states that everything is clean as in \emph{not common} by itself. As translated in YLT:
``I have known, and am persuaded, in the Lord Jesus, that nothing is unclean of itself, except to him who is reckoning
anything to be unclean -- to that one it is unclean.'' It is a very strong doctrine that resonates similarly with what
Jesus says to the blind man in Matt 9:29 (YLT) ``According to your faith let it be to you.'' If you make up a faith that
declares something as common to you, then it will be so. It is better to realize that there is nothing that ever is
common to us except what we live out in our lives and about when we start to speak it out from our heart.
Even in Rev 21:27 it says that nothing common will ever enter into the gates of the Heavenly Jerusalem. Therefore it is
best to realize that if you are a Gentile, 1) that you are not common, refuse to see others as common and thereby
commonize them. 2) Realize that there are nothing that in itself is common and start practicing that by ridding oneself
of false uncleanliness doctrines. And 3) to stop practice, and even more not speak the commonizing words of substandard
heart reasonings.


%\grc{κοινόω}
%mat 15:11; mat 15:18; mat 15:20; mar 7:15; mar 7:18; mar 7:20; mar 7:23; act 10:15; act 11:9; act 21:28; heb 9:13

%\grc{κοινός}
%mar 7:2; mar 7:5; act 2:44; act 4:32; act 10:14; act 10:28; act 11:8; \emph{rom 14:14}; tit 1:4; heb 10:29; jud 1:3; rev 21:27
    \item[Decent,]
\entlbl{decent}

\grc{ἁγνός, ἁγνεία, ἁγνότης}
\index[grc]{βδελυσσομαι@\grc{βδελύσσομαι}}
(\textit{hagnos, hagneia, hagnotēs}):
\newglossaryentry{hagnos, hagneia, hagnotēs}
{
    name=\grc{ἁγνός, ἁγνεία, ἁγνότης},
    description={\entrefgls{decent}},
    sort=βδελυσσομαι@\grc{ἁγνός, ἁγνεία, ἁγνότης}
}


    \item[Vulgarity,]
\entlbl{vulgarity}

\grc{ῥυπαρία}
\index[grc]{βδελυσσομαι@\grc{βδελύσσομαι}}
(\textit{rhyparia}):
\newglossaryentry{rhyparia}
{
    name=\grc{ῥυπαρία},
    description={\entrefgls{vulgarity}},
    sort=βδελυσσομαι@\grc{ῥυπαρία}
}
Is mentioned in Jas 1:21 and is only used once, therefore no other context in the Bible may be used to define it, 
as a result the only source of interpretation are available dictionaries, a new interpretation is the one used by 
BW10 which is \emph{vulgarity}. 

\end{description}

%\input{section/all_lists}
%\input{section/vice_index}

\section{Conclusion}

\section{Bibliography}
% \begin{description}
	% \item E. Nestle, K. Aland et al., \emph{Novum Testamentum Graece, twenty-eighth edition}. Stuttgart: Deutsche Bibelgesellschaft, 2012.
	% \item Bushell, Michael S., Jean-Noel Aletti and Andrzej Gieniusz. \emph{BibleWorks Greek New Testament Morphology (BNM)}. 1999-2001. \emph{BibleWorks 10.0.8.755} BibleWorks, 2017.
	% \item Kittel, Gerhard, and Gerhard Friedrich, eds. \emph{Theological Dictionary of the New Testament}. Translated by Geoffrey W. Bromiley. 10 vols. Grand Rapids: Eerdmans, 1964--1976.
	% \item Henry George Liddell. Robert Scott.  \emph{A Greek-English Lexicon}. Revised and augmented throughout by Sir Henry Stuart Jones, with the assistance of Roderick McKenzie. Oxford: Clarendon Press, 1940.
	% \item Thayer, Joseph Henry. \emph{Greek-English Lexicon of the New Testament}. 4th ed. Edinburgh: T. \& T. Clark, 1896.
	% \item Loder, Charles. \emph{Greek Transliteration}. (Internet). 2022. \emph{greek-transliteration v2.0.0} Greek Transliteration, 2023. https://charlesloder.github.io\\/greekTransliteration/index.html
% \end{description}

\onecolumn
% \printindex[vicelist]
% \printindex[eng]
%\printindex[grc]

\end{document}
