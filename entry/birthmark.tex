\item[Birthmark,]
\entlbl{decent}

\grc{σπίλος, σπιλόω, ἄσπιλος}
\index[grc]{βδελυσσομαι@\grc{βδελύσσομαι}}
(\textit{spilos, spiloō, aspilos}):
\newglossaryentry{spilos, spiloō, aspilos}
{
    name=\grc{σπίλος, σπιλόω, ἄσπιλος},
    description={\entrefgls{mole}},
    sort=βδελυσσομαι@\grc{spilos, spiloō, aspilos}
} At a first glance it's impossible to understand the meanining of \grc{σπίλος}, 
according to Liddell it means \emph{a spot}, \emph{stain}, \emph{blemish} but that's all
that is mentioned. In context of the Bible, \grc{σπίλος} together with related cognates 
are used only 8 times. Also other religious Greek dictionaries give a similar insufficient
material to fully understand its true meaning. Only when we turn to modern Greek we get a
hint of what it actually means. The modern meaning is 
\emph{naevus}\footnote{\emph{TechDico}, s.v. ``\grc{σπίλος},'' accessed 2023--09--22,\\https://www.techdico.com/translation/greek-english/\grc{σπίλος}.html.},
which is presumably a strawberry birthmark\footnote{\emph{Oxford Reference}, s.v. ``naevus,'' accessed 2023--09--22,\\https://www.oxfordreference.com/search?q=naevus.}, 
possibly it could spread such as skincancer. 
Furthermore, Paul writes that Jesus wishes to ``present it to himself the assembly in glory, not having \emph{spot} [emphasis added] or wrinkle'' Eph 5:27 YLT, in this verse
the spot which is described, is a very strawberry birthmark. Also in 1 Tim 6:14 Paul admonishes Timothy in Christ Jesus to ``keep the command \emph{unspotted} [emphasis added], 
unblameable, till the manifestation of our Lord Jesus Christ,'' YLT. In this very context Paul had given a commandment to keep in v.11. Totally there are two ways to be
blemished with spiritual strawberry birthmarks in the NT. The first one is in Jas 3:6 saying that the tongue is the member that produces naevus over our whole body and is
ignited by the fire of hell. The naevus in this case should be understood spiritually and the passage should be understood properly by reading the full context.
The second contamination by strawberry birthmarks is mentioned in Jude 1:23, in this context we are admonished to show grace to those who need to be snatched out of the
fire, while being disgusted by the undergarment being blemished by the flesh, or likely the \emph{meat}. The context is very advanced and should be thoroughly
studied. The key components could be found in verse 7--8, presumably connections to sex purchase, unnatural meat and defilment in sleep with visions and nocturnal emissions.
In 2 Pet 3:14 there is a solution for those that has been blemished by the aforementioned behaviors that caused naevus. ``Wherefore beloved, while await you shall speed to be 
unblemished and blameless at piece in him (NA28; my translation).'' This gives the sinners a strong encouragment to find a cure in Christ before he comes back. While they remain
in the world, the only way to keep unblemished from the naevus of the world, are to visit widows and orphans in their tribulation as the only true religion, in accordance with Jas 1:27.
The last thing to be said is that the Blood of Christ with which we were bought is without naevus (1 Pet 1:19).
