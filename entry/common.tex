\item[Common, commonize,]
\entlbl{common, commonize}

\grc{κοινός, κοινόω}
\index[grc]{βδελυσσομαι@\grc{βδελύσσομαι}}
(\textit{koinos, koinoō}):
\newglossaryentry{koinos, koinoō}
{
    name=\grc{κοινός, κοινόω},
    description={\entrefgls{common, commonize}},
    sort=βδελυσσομαι@\grc{κοινός, κοινόω}
}
In examining what type of cleanliness or uncleanliness that \grc{κοινός} is used for as a biblical synonym it must be
understood how it was used carnally by the Pharisees. Both in Matt 15 and in Mark 7, the Pharisees had a serious
argument why Jesus allowed the disciples to eat with \emph{unwashed hands}, that is hands that had been commonized by being
in physical touch with what was common.
Further the religious Jews accused Paul to bring Gentiles into the Temple in Jerusalem and thereby commonizing the holy
site and starting a riot, as mentioned in Acts 21:28.
Also in Acts chapter 10--11 Peter is guided by God to believe that what was unsanitary had been declared clean, no
longer commonized, as in Acts 10:28 ``But God has shown me to not call any human \emph{common} or \emph{unsanitary}
(NA28; my translation).'' God did also clearly state this in 10:15b ``Those who God \emph{sanitized}, you shall not \emph{commonize} (NA28; my translation).''
What is said in Mark 7:18--19 and Matt 15:16--18 is all about uncleanliness, Jesus clearly states that what you eat and later waste is
not what makes one common. What really makes one common is what comes out from the heart either in
actions or as encouraging babble. That which is mentioned are two sets of vices from parallel passages but with somewhat
different sins mentioned, and therefore, some sins may require to be talked about in order to make common.
The two sets of vices are:

%\begin{table}[]
%\caption{}
% \label{tab:my-table}
\begin{tabular}{lll}
& Matt 15:19 & Mark 7:21--22 \\
\grc{ψευδομαρτυρία} & \hfil \checkmark &  \\
\grc{πορνεία} & \hfil \checkmark & \hfil \checkmark \\
\grc{κλοπή} & \hfil \checkmark & \hfil \checkmark \\
\grc{φόνος} & \hfil \checkmark & \hfil \checkmark \\
\grc{μοιχεία} & \hfil \checkmark & \hfil \checkmark \\
\grc{πλεονεξία} &  & \hfil \checkmark \\
\grc{πονηρία} &  & \hfil \checkmark \\
\grc{δόλος} &  & \hfil \checkmark \\
\grc{ἀσέλγεια} &  & \hfil \checkmark \\
\grc{ὀφθαλμός πονηρός} &  & \hfil \checkmark \\
\grc{βλασφημία} &  & \hfil \checkmark \\
\grc{ὑπερηφανία} & \hfil \checkmark & \hfil \checkmark \\
\grc{ἀφροσύνη} &  & \hfil \checkmark
\end{tabular}
%\end{table}

Mark 7:15 mentions that it is what comes out of the person which defiles them, both in Matthew and in Mark according
to their respective context makes clear that it is only what comes out from the heart that defiles men and, in
Matt 15:10 specifically what comes out through the mouth, considered to be verbatim, that commonizes that person.
Simply said, it is when the person starts to speak according to the substandard reasonings of their heart, when
they actually become common.
Moreover, in Rom 14:14 we find a rule that states that everything is clean as in \emph{not common} by itself. As translated in YLT:
``I have known, and am persuaded, in the Lord Jesus, that nothing is unclean of itself, except to him who is reckoning
anything to be unclean -- to that one it is unclean.'' It is a very strong doctrine that resonates similarly with what
Jesus says to the blind man in Matt 9:29 (YLT) ``According to your faith let it be to you.'' If you make up a faith that
declares something as common to you, then it will be so. It is better to realize that there is nothing that ever is
common to us except what we live out in our lives and about when we start to speak it out from our heart.
Even in Rev 21:27 it says that nothing common will ever enter into the gates of the Heavenly Jerusalem. Therefore it is
best to realize that if you are a Gentile, 1) that you are not common, refuse to see others as common and thereby
commonize them. 2) Realize that there are nothing that in itself is common and start practicing that by ridding oneself
of false uncleanliness doctrines. And 3) to stop practice, and even more not speak the commonizing words of substandard
heart reasonings.


%\grc{κοινόω}
%mat 15:11; mat 15:18; mat 15:20; mar 7:15; mar 7:18; mar 7:20; mar 7:23; act 10:15; act 11:9; act 21:28; heb 9:13

%\grc{κοινός}
%mar 7:2; mar 7:5; act 2:44; act 4:32; act 10:14; act 10:28; act 11:8; \emph{rom 14:14}; tit 1:4; heb 10:29; jud 1:3; rev 21:27